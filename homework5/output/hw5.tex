\documentclass{article}
\usepackage[utf8]{inputenc}
\usepackage{hyperref}
\usepackage[letterpaper, portrait, margin=1in]{geometry}
\usepackage{enumitem}
\usepackage{amsmath}

\usepackage{titlesec}

\titleformat{\section}
{\normalfont\Large\bfseries}{\thesection}{1em}{}[{\titlerule[0.8pt]}]
  
\title{Homework 5}
\author{Economics 7103}
\date{ }
  
\begin{document}

\maketitle

\noindent You have access to imaginary vehicle sales data from 2017 (\textit{instrumentalvehicles.csv}).  You are interested in estimating the hedonic price of an additional mile per gallon as part of a larger analysis of willingness to pay for fuel efficiency.  In your data, you have the following variables:
\begin{table}[h]
    \centering
    \begin{tabular}{l|l}
        Variable & Description \\ \hline
         \textit{price} & Sales price of the vehicle in USD  \\
         \textit{car} & Class of the vehicle. =1 if the vehicle is a sedan, =0 if the vehicle is an SUV \\
         \textit{mpg} & Fuel efficiency in miles per gallon \\
         \textit{weight} & Weight of the vehicle in pounds \\
         \textit{height} & Height of the vehicle in inches \\
         \textit{length} & Length of the vehicle in inches \\
    \end{tabular}
    \caption{Variable descriptions for homework 6.}
    \label{tab:variables6}
\end{table}

\section{Python}

\begin{enumerate}
    \item Run the ordinary-least-squares regression of price on \textit{mpg}, the \textit{car} indicator variable, and a constant.  Report and interpret the coefficient on miles per gallon (do not construct a table).
    \item What forms of endogeneity are you concerned about when estimating the coefficient on \textit{mpg}?
    \item To correct for this endogeneity, you would like to use instrumental variables.  Specifically, you are interested in the system of equations:
    \begin{align}
        price_v &= \beta_0 + \beta_1 mpg_v + \beta_2 car_v + e_v \\
        mpg_v &= \gamma_0 + \gamma_1 z_v +\gamma_2 car_v + u_v,
    \end{align}
    where $z_v$ is the value of the instrument for vehicle $v$ and $e_v$ and $u_v$ are error terms.  Report the estimated second-stage coefficients, standard errors or confidence intervals, and the first-stage $F$-statistic for the excluded instrument in the same table for the following procedures (just us a regular F-statistic for this exercise rather than the robust Montiel-Olea-Pflueger F-statistic):
    \begin{enumerate} [label=(\alph*)]
        \item Perform two-stage-least-squares estimation by hand using $weight$ as the excluded instrument.  (First regress $mpg$ on all of the instruments.  Save the fitted values from the first stage $\hat{mpg}$ and use the fitted values in place of the endogenous variable in the second stage price regression.)  
        \item Perform two-stage-least-squares estimation by hand using $weight^2$ as the excluded instrument.
        \item Perform two-stage-least-squares estimation by hand using $height$ as the excluded instrument.
        \item In words, what are the different exclusion restrictions required for parts (a)-(c)?  Does this seem reasonable for these instruments?
        \item Compare and contrast the estimated coefficient on $mpg$ from parts (a)-(c).  What explains the discrepancies?
    \end{enumerate}
    \item Calculate the IV estimate using GMM with $weight$ as the excluded instrument. (Look for the Linearmodels function \verb!IVGMM!).  Report the estimated second-stage coefficient and standard error or confidence interval for $mpg$  What factors account for the differences in the standard errors?
\end{enumerate}

\section{Stata}

\begin{enumerate}
    \item Use the \verb|ivregress liml| command to compute the limited information maximum likelihood estimate using weight as the excluded instrument.  Report your second-stage results in a nicely-formatted table using \verb|outreg2|.  Use heteroskedasticity-robust standard errors.
    \item Use \verb|weakivtest| to estimate the Montiel-Olea-Pflueger effective F-statistic.  What is the 5\% critical value, the F-statistic, and conclusion?
\end{enumerate}



\end{document}